\chapter{Zaključak i budući rad}
		Naš projektni zadatak bio je realizacija web aplikacije Sinappsa koja omogućuje studentima FER-a da traže ili pružaju pomoć vezanu za fakultet. Nakon skoro cijelog semestra rada na aplikaciji ponosno možemo reći da smo implementirali i zadovoljili sve zahtjeve radnog zadatka.
		
		Prvih sedam tjedana rada na aplikaciji išli su sporijim tempom. Prije početka samog programiranja aplikacije, proučili smo projektni zadatak, napravili obrasce uporabe te se dogovorili kako će aplikacija izgledati i od kojih komponenata će se sastojati. Zatim smo se podijelili na frontend, backend i dokumentaciju. Nakon podjele krenuli smo se upoznavati s tehnologijama s kojima je većina nas prvi put stupila u kontakt. Prije prve revizije imali smo osposobljene osnovne funkcionalnosti aplikacije, a to su bile registracija, prijava i dizajn stranice.
		
		Nakon prve revizije, u dva tjedna napravljene su gotovo sve funkcionalnosti aplikacije kako bi se mogla predati alfa inačica aplikacije. Nakon manjih prepravaka u kodu, aplikacija je bila gotova i spremna za deploy. Rad na dokumentaciji prvenstveno se sastojao od crtanja i opisivanja UML dijagrama arhitekture aplikacije. Na kraju je učinjeno ispitivanje sustava i komponenti kako bi se provjerio rad aplikacije. 
		
		Uz komunikaciju preko WhatsApp-a, skoro smo svaki tjedan imali sastanke uživo ili preko Microsoft Teamsa kako bi rasporedili zadatke za nadolazeći tjedan. Česti sastanci natjerali su nas da radimo redovito te su poticali što bolju komunikaciju među članovima tima.
		Rad na ovom projektu bio je iznimno koristan jer nam je dao uvid u stvarni svijet web developera, od dizajniranja stranice do izrade dokumentacije. Također je prikazao važnost harmonijskog rada u grupi kako bi sve bilo izrađeno i predano na vrijeme. Iako u određenim trenutcima težak, ovaj projekt je bio veoma bitan kao primjer simulacije rada u industriji.
		
		\eject 
